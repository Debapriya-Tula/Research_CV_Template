  %%%%%%%%%%%%%%%%%%%%%%%%%%%%%%%%%%%%%%%%%%%%%%%%%%%%%%%%%%%%%%%%%%%%%%%%
%%%%%%%%%%%%%%%%%%%%%% Simple LaTeX CV Template %%%%%%%%%%%%%%%%%%%%%%%%
%%%%%%%%%%%%%%%%%%%%%%%%%%%%%%%%%%%%%%%%%%%%%%%%%%%%%%%%%%%%%%%%%%%%%%%%

%%%%%%%%%%%%%%%%%%%%%%%%%%%%%%%%%%%%%%%%%%%%%%%%%%%%%%%%%%%%%%%%%%%%%%%%
%% NOTE: If you find that it says                                     %%
%%                                                                    %%
%%                           1 of ??                                  %%
%%    
%%

%% at the bottom of your first page, this means that the AUX file     %%
%% was not available when you ran LaTeX on this source. Simply RERUN  %%
%% LaTeX to get the ``??'' replaced with the number of the last page  %%
%% of the document. The AUX file will be generated on the first run   %%
%% of LaTeX and used on the second run to fill in all of the          %%
%% references.                                                        %%
%%%%%%%%%%%%%%%%%%%%%%%%%%%%%%%%%%%%%%%%%%%%%%%%%%%%%%%%%%%%%%%%%%%%%%%%

%%%%%%%%%%%%%%%%%%%%%%%%%%%% Document Setup %%%%%%%%%%%%%%%%%%%%%%%%%%%%

% Don't like 10pt? Try 11pt or 12pt
\documentclass[10pt]{article}

% The automated optical recognition software used to digitize resume
% information works best with fonts that do not have serifs. This
% command uses a sans serif font throughout. Uncomment both lines (or at
% least the second) to restore a Roman font (i.e., a font with serifs).
%\usepackage{times}
%\renewcommand{\familydefault}{\sfdefault}

% This is a helpful package that puts math inside length specifications
\usepackage{calc}
\usepackage{comment}
\usepackage{url}
\usepackage[T1]{fontenc}
% For "lorem ipsum" text
\usepackage[english]{babel}
\usepackage{lipsum}
\usepackage{multicol}
\usepackage{pifont}
\usepackage{fontawesome}
\usepackage{etaremune}
% Simpler bibsection for CV sections
% (thanks to natbib for inspiration)
\makeatletter
\newlength{\bibhang}
\setlength{\bibhang}{1em} %1em}
\newlength{\bibsep}
 {\@listi \global\bibsep\itemsep \global\advance\bibsep by\parsep}
\newenvironment{bibsection}%
        {\begin{enumerate}{}{%
%        {\begin{list}{}{%
       \setlength{\leftmargin}{\bibhang}%
       \setlength{\itemindent}{-\leftmargin}%
       \setlength{\itemsep}{\bibsep}%
       \setlength{\parsep}{\z@}%
        \setlength{\partopsep}{0pt}%
        \setlength{\topsep}{0pt}}}
        {\end{enumerate}\vspace{-.6\baselineskip}}
%        {\end{list}\vspace{-.6\baselineskip}}
\makeatother

\usepackage{tikz}

\newcommand{\ExternalLink}{%
    \tikz[x=1.2ex, y=1.2ex, baseline=-0.05ex]{% 
        \begin{scope}[x=1ex, y=1ex]
            \clip (-0.1,-0.1) 
                --++ (-0, 1.2) 
                --++ (0.6, 0) 
                --++ (0, -0.6) 
                --++ (0.6, 0) 
                --++ (0, -1);
            \path[draw, 
                line width = 0.5, 
                rounded corners=0.5] 
                (0,0) rectangle (1,1);
        \end{scope}
        \path[draw, line width = 0.5] (0.5, 0.5) 
            -- (1, 1);
        \path[draw, line width = 0.5] (0.6, 1) 
            -- (1, 1) -- (1, 0.6);
        }
    }

% Layout: Puts the section titles on left side of page
% \reversemarginpar

%
%         PAPER SIZE, PAGE NUMBER, AND DOCUMENT LAYOUT NOTES:
%
% The next \usepackage line changes the layout for CV style section
% headings as marginal notes. It also sets up the paper size as either
% letter or A4. By default, letter was used. If A4 paper is desired,
% comment out the letterpaper lines and uncomment the a4paper lines.
%
% As you can see, the margin widths and section title widths can be
% easily adjusted.
%
% ALSO: Notice that the includefoot option can be commented OUT in order
% to put the PAGE NUMBER *IN* the bottom margin. This will make the
% effective text area larger.
%
% IF YOU WISH TO REMOVE THE ``of LASTPAGE'' next to each page number,
% see the note about the +LP and -LP lines below. Comment out the +LP
% and uncomment the -LP.
%
% IF YOU WISH TO REMOVE PAGE NUMBERS, be sure that the includefoot line
% is uncommented and ALSO uncomment the \pagestyle{empty} a few lines
% below.
%
% https://www.overleaf.com/project/615c98f1a13a742fb3e04950
%% Use these lines for letter-sized paper
%\usepackage[paper=letterpaper,
%            %includefoot, % Uncomment to put page number above margin
%            marginparwidth=1.2in,     % Length of section titles
%            marginparsep=.05in,       % Space between titles and text
%            margin=1in,               % 1 inch margins
%            includemp]{geometry}

%% Use these lines for A4-sized paper
% \usepackage[paper=a4paper,
%             %includefoot, % Uncomment to put page number above margin
%             marginparwidth=27.5mm,    % Length of section titles
%             marginparsep=1.5mm,       % Space between titles and text
%             margin=18mm,              % 25mm margins
%             includemp]{geometry}

\usepackage[paper=a4paper,
            includefoot, % Uncomment to put page number above margin
            marginparwidth=0mm,    % Length of section titles
            marginparsep=0mm,       % Space between titles and text
            margin=10mm,              % 25mm margins
            includemp]{geometry}

%% More layout: Get rid of indenting throughout entire document
\setlength{\parindent}{0in}

\usepackage[shortlabels]{enumitem}

%% Reference the last page in the page number
%
% NOTE: comment the +LP line and uncomment the -LP line to have page
%       numbers without the ``of ##'' last page reference)
%
% NOTE: uncomment the \pagestyle{empty} line to get rid of all page
%       numbers (make sure includefoot is commented out above)
%
\usepackage{fancyhdr,lastpage}
\pagestyle{fancy}
% \pagestyle{empty}      % Uncomment this to get rid of page numbers
\fancyhf{}\renewcommand{\headrulewidth}{0pt}
\fancyfootoffset{\marginparsep+\marginparwidth}
\newlength{\footpageshift}
\setlength{\footpageshift}
          {0.5\textwidth+0.5\marginparsep+0.5\marginparwidth-2in}
\lfoot{\hspace{\footpageshift}%
       \parbox{4in}{\, \hfill %
                    \arabic{page} of \protect\pageref*{LastPage} % +LP
%                    \arabic{page}                               % -LP
                    \hfill \,}}

% Finally, give us PDF bookmarks
\usepackage{color,hyperref}
\definecolor{darkblue}{rgb}{0.0,0.0,0.3}
\hypersetup{colorlinks,breaklinks,
            linkcolor=darkblue,urlcolor=darkblue,
            anchorcolor=darkblue,citecolor=darkblue}

%%%%%%%%%%%%%%%%%%%%%%%% End Document Setup %%%%%%%%%%%%%%%%%%%%%%%%%%%%


%%%%%%%%%%%%%%%%%%%%%%%%%%% Helper Commands %%%%%%%%%%%%%%%%%%%%%%%%%%%%

% The title (name) with a horizontal rule under it
% (optional argument typesets an object right-justified across from name
%  as well)
%
% Usage: \makeheading{name}
%        OR
%        \makeheading[right_object]{name}
%
% Place at top of document. It should be the first thing.
% If ``right_object'' is provided in the square-braced optional
% argument, it will be right justified on the same line as ``name'' at
% the top of the CV. For example:
%
%       \makeheading[\emph{Curriculum vitae}]{Your Name}
%
% will put an emphasized ``Curriculum vitae'' at the top of the document
% as a title. Likewise, a picture could be included:
%
%   \makeheading[\includegraphics[height=1.5in]{my_picutre}]{Your Name}
%
% the picture will be flush right across from the name.
\newcommand{\makeheading}[2][]%
        {\hspace*{-\marginparsep minus \marginparwidth}%
         \begin{minipage}[t]{\textwidth+\marginparwidth+\marginparsep}%
             {\large \bfseries #2 \hfill #1}\\[-0.15\baselineskip]%
                 \rule{\columnwidth}{1pt}%
         \end{minipage}}
% \newcommand{\makeheading}[2][]%
%         {#1 \hfill #2}

% The section headings
%
% Usage: \section{section name}
% \renewcommand{\section}[1]{\pagebreak[3]%
%     \hyphenpenalty=10000%
%     \vspace{1.3\baselineskip}%
%     \phantomsection\addcontentsline{toc}{section}{#1}%
%     \noindent\llap{\scshape\smash{\parbox[t]{\marginparwidth}{\raggedright #1}}}%
%     \vspace{-\baselineskip}\par}

    
\renewcommand{\section}[1]{
\bigskip
  \begin{Large}
  {\textsc{\textbf{#1}}}
  \end{Large}
  \hrulefill
  \medskip
  \\
}

% An itemize-style list with lots of space between items
\newenvironment{outerlist}[1][\enskip\textbullet]%
        {\begin{itemize}[#1,leftmargin=*]}{\end{itemize}%
         \vspace{-.6\baselineskip}}

% An environment IDENTICAL to outerlist that has better pre-list spacing
% when used as the first thing in a \section
\newenvironment{lonelist}[1][\enskip\textbullet]%
        {\begin{list}{#1}{%
        \setlength{\partopsep}{0pt}%
        \setlength{\topsep}{0pt}}}
        {\end{list}\vspace{-.6\baselineskip}}

% An itemize-style list with little space between items
\newenvironment{innerlist}[1][\enskip\textbullet]%
        {\begin{itemize}[#1,leftmargin=*,parsep=0pt,itemsep=0pt,topsep=0pt,partopsep=0pt]}
        {\end{itemize}\vspace{-.1\baselineskip}}

% An environment IDENTICAL to innerlist that has better pre-list spacing
% when used as the first thing in a \section
\newenvironment{loneinnerlist}[1][\enskip\textbullet]%
        {\begin{itemize}[#1,leftmargin=*,parsep=0pt,itemsep=0pt,topsep=0pt,partopsep=0pt]}
        {\end{itemize}\vspace{-.6\baselineskip}}

        
% An itemize-style list with medium space between items
\newenvironment{midlist}[1][\enskip\textbullet]%
        {\begin{itemize}[#1,leftmargin=*,parsep=0pt,itemsep=1.9pt,topsep=0pt,partopsep=0pt, ]}
        {\end{itemize}}

% An itemize-style list with small space between items
\newenvironment{smallmidlist}[1][\enskip\textbullet]%
        {\begin{itemize}[#1,leftmargin=*,parsep=0pt,itemsep=1pt,topsep=0pt,partopsep=0pt]}
        {\end{itemize}}
        
% An environment IDENTICAL to midlist that has better pre-list spacing
% when used as the first thing in a \section
\newenvironment{lonemidlist}[1][\enskip\textbullet]%
        {\begin{itemize}[#1,leftmargin=*,parsep=0pt,itemsep=4pt,topsep=0pt,partopsep=0pt]}
        {\end{itemize}}


% To add some paragraph space between lines.
% This also tells LaTeX to preferably break a page on one of these gaps
% if there is a needed pagebreak nearby.
\newcommand{\blankline}{\quad\pagebreak[3]}
\newcommand{\halfblankline}{\quad\vspace{-0.5\baselineskip}\pagebreak[3]}

% Uses hyperref to link DOI
\newcommand\doilink[1]{\href{http://dx.doi.org/#1}{#1}}
\newcommand\doi[1]{doi:\doilink{#1}}

% For \url{SOME_URL}, links SOME_URL to the url SOME_URL
\providecommand*\url[1]{\href{#1}{#1}}
% Same as above, but pretty-prints SOME_URL in teletype fixed-width font
\renewcommand*\url[1]{\href{#1}{\texttt{#1}}}

% For \email{ADDRESS}, links ADDRESS to the url mailto:ADDRESS
\providecommand*\email[1]{\href{mailto:#1}{#1}}
% Same as above, but pretty-prints ADDRESS in teletype fixed-width font
%\renewcommand*\email[1]{\href{mailto:#1}{\texttt{#1}}}

%\providecommand\BibTeX{{\rm B\kern-.05em{\sc i\kern-.025em b}\kern-.08em
%    T\kern-.1667em\lower.7ex\hbox{E}\kern-.125emX}}
%\providecommand\BibTeX{{\rm B\kern-.05em{\sc i\kern-.025em b}\kern-.08em
%    \TeX}}
\providecommand\BibTeX{{B\kern-.05em{\sc i\kern-.025em b}\kern-.08em
    \TeX}}
\providecommand\Matlab{\textsc{Matlab}}
\newcommand{\homepage}{https://debapriya-tula.github.io/}
%%%%%%%%%%%%%%%%%%%%%%%% End Helper Commands %%%%%%%%%%%%%%%%%%%%%%%%%%%

%%%%%%%%%%%%%%%%%%%%%%%%% Begin CV Document %%%%%%%%%%%%%%%%%%%%%%%%%%%%

\begin{document}
\begin{LARGE}
\begin{center}\textbf{Debapriya Tula}\end{center}
\end{LARGE}
\smallskip 
\faEnvelope~\email{dtula@g.ucla.edu} \hfill
\faHome~\href{\homepage}{debapriya-tula.github.io} \hfill
\faLinkedin~\href{http://linkedin.com/in/debapriya-tula}{LinkedIn} \hfill
\faGraduationCap~\href{https://scholar.google.com/citations?view_op=list_works&hl=en&authuser=3&user=4lJhtPYAAAAJ}{Google Scholar} \hfill
\faGithub~\href{http://github.com/Debapriya-Tula}{GitHub} \hfill


\section{Education}
\textbf{University of California, Los Angeles} \hfill {\textit{2024 - 2026}}\\
\vspace{1mm}\emph{Masters in Electrical and Computer Engineering}\\
\textbf{Indian Institute of Information Technology, Sri City} \hfill {\textit{2017 - 2021}}\\
\emph{Bachelor of Technology in Computer Science and Engineering} \hfill{GPA: $\textbf{9.35}/10.0$}
        % Advisors: \href{https://homes.cs.washington.edu/~ali/}{Prof. Ali Farhadi} \& \href{https://homes.cs.washington.edu/~sham/}{Prof. Sham Kakade}\\

% \textbf{Indian Institute of Technology Bombay} \hfill {\textit{2013 - 2017}}\\
% \emph{B.Tech (Honours) in Computer Science and Engineering with Minor in Electrical Engineering}\\%\hfill{GPA: 8.63/10}
%         Advisor: \href{http://www.cse.iitb.ac.in/~soumen/}{Prof. Soumen Chakrabarti}

\vspace{-1.0mm}
\section{Research Experience}
\textbf{Google Deepmind, India} \hfill {\textit{Aug 2022 - Aug 2024}}\\
\emph{Pre-Doctoral Researcher}\\
Advisors: Dr. Prateek Jain \href{https://www.prateekjain.org/}{\ExternalLink} \& {Dr. Sujoy Paul} \href{https://research.google/people/107637/}{\ExternalLink}  \\[2.3pt]
\underline{Test-time adaptation of OCR models.} \vspace{1mm}
\begin{midlist}
    \item Formulated the novel problem of test-time adaptation of OCR models to a \textbf{single} image of a writer's handwriting.
    \item Designed a confidence and consistency based self-training method to improve predictions over the single image iteratively.
    \item Improved CER over \textbf{0.4} \% across \textbf{250+} internal datasets. 
\end{midlist} \vspace{1.5mm}

\underline{Streamlined encoding of text-embedded images for efficient vision-language models.} \vspace{1mm}
\begin{midlist}
    \item Developed a method to embed textual content of an image within the image, for direct processing by vision encoders.
    \item Achieved \textbf{twofold} improvement in exact match scores compared to baseline method without embedded textual content.
    % \item Experimenting with a variety of vision-language tasks to check the efficacy and adaptability of the proposed method.
\end{midlist}
\vspace{1.5mm}

\underline{Efficiency in video and image generation} \vspace{1mm}
\begin{midlist}
    \item Implemented Matryoshka learning with distillation for transformer layers, to decrease inference latency.
    \item Performed exhaustive ablations of Matryoshka over transformer layers.
\end{midlist}
\vspace{2.0mm}


\textbf{IIT Delhi, India} \hfill {\textit{May 2020 - July 2020}}\\
\emph{Computer Vision Research Intern} \\
Advisor: Dr. Brejesh Lall \href{https://in.linkedin.com/in/gauagg}{\ExternalLink}
\begin{midlist}
    \item Designed an efficient pipeline for the problem of motion segmentation of fish in \textbf{underwater scenarios} solved as an \textbf{unsupervised} learning task.
    \item Modelled underwater disturbances and designed a temporal autoencoder based pipeline for the problem.
    \item Implemented over 4 \textbf{video object segmentation} papers to assess their transferability to this challenging setting.
\end{midlist}
\vspace{2.0mm}

\textbf{Tezpur University, India} \hfill {\textit{May 2019 - June 2019}}\\
\emph{Research Intern} \\
Advisor: Dr. Siddhartha S. Satapathy \href{https://tezu.irins.org/profile/48195}{\ExternalLink}
\begin{midlist}
    \item Developed an algorithm for maximizing stacking regions to estimate most stable secondary structures for RNA sequences.
    \item Designed a dynamic programming based solution and used graph concepts like maximum independent sets and circle graphs to simplify the problem.
    \item Awarded the \textbf{best paper} at \textbf{ICCCIoT, 2020}.
\end{midlist}
% \vspace{-mm}

% \section{Publications}
% \label{sec:pubs}
% \textbf{\large{Preprints}}
% % \vspace{-\topsep}
% \begin{itemize} [topsep=2pt] \itemsep-0.07em
%     \item \textbf{Is it an i or an l: Test-time Adaptation of Text Line Recognition Models.} \href{https://arxiv.org/abs/2308.15037}{\ExternalLink}\\
%     \textbf{D Tula}, S Paul, G Madan, P Garst, R Ingle, G Aggarwal. \\
%     \emph{Under review, 2023}
% \end{itemize}

% \textbf{\large{Published}}
% % \vspace{-\topsep}
% \begin{itemize} [topsep=2pt] \itemsep-0.07em
%     \item \textbf{Target Aware Network Architecture Search and Compression for Efficient Knowledge Transfer.} \href{https://arxiv.org/abs/2205.05967}{\ExternalLink}\\
%     S Basha, \textbf{D Tula}, S Vinakota, S R Dubey. \\
%     \emph{Multimedia Systems, 2024 (Journal)}.
    
%     \item \textbf{Offense Detection in Dravidian Languages using Code-Mixing Index based Focal Loss and Cosine Normalization.} \href{https://link.springer.com/article/10.1007/s42979-022-01190-1}{\ExternalLink}\\
%     \textbf{D Tula}, Shreyas Ms, V Reddy, P Sahu, S Doddapaneni, P Potluri, R Sukumaran, P Patwa.\\
%     \emph{Springer Nature Computer Science, 2022 (Journal)}.
    
%     \item \textbf{Ensemble of Multilingual Language Models with Pseudo Labeling for Offense Detection in Dravidian Languages.} \href{https://www.aclweb.org/anthology/2021.dravidianlangtech-1.42}{\ExternalLink}\\
%     \textbf{D Tula}, P Potluri, Shreyas Ms, S Doddapaneni, P Sahu, R Sukumaran, P Patwa.\\
%     \emph{European Association for Computational Linguistics (\textbf{EACL}), 2021}\\
%     \emph{DravidianLangTech workshop, 2021}.
    
%     \item \textbf{Estimating RNA Secondary Structure by Maximizing Stacking Regions.} \href{https://doi.org/10.1007/978-981-15-6198-6_15}{\ExternalLink}\\
%     P Sen, \textbf{D Tula}, S K Ray, S S Satapathy. \\
%     \emph{International Conference on Computer Communication and Internet of Things (\textbf{ICCCIoT}), 2021.} \\
%     \faTrophy \textcolor{black}{~\textbf{Best Paper Award}.}
    
%     % \item \href{https://pubmed.ncbi.nlm.nih.gov/35920776/}{\textbf{Incorporation of transition to transversion ratio and nonsense mutations, improves the estimation of the number of synonymous and non-synonymous sites in codons.}}\\
%     % Suvendra K Ray, Ruksana Aziz, Piyali Sen, Pratyush Kumar Beura, Saurav Das, \textbf{Debapriya Tula}, \\
%     % Madhusmita Dash, Nima Dondu Namsa, Ramesh Chandra Deka, Edward J Feil, Siddhartha Sankar Satapathy.\\
%     % \emph{DNA Research, 2022 (Journal)}.

% \end{itemize}
% \vspace{-1.5mm}


\section{Publications}
\label{sec:pubs}
\vspace{-\baselineskip}
\vspace{-1.0mm}
\begin{itemize} [topsep=1.5pt] \itemsep-0.09em
    \item \textbf{Target Aware Network Architecture Search and Compression for Efficient Knowledge Transfer.} \href{https://arxiv.org/abs/2205.05967}{\ExternalLink}\\
    S Basha, \textbf{D Tula}, S Vinakota, S R Dubey. \\
    \emph{Multimedia Systems, 2024 (Journal)}.

    \item \textbf{Is it an i or an l: Test-time Adaptation of Text Line Recognition Models.} \href{https://arxiv.org/abs/2308.15037}{\ExternalLink}\\
    \textbf{D Tula}, S Paul, G Madan, P Garst, R Ingle, G Aggarwal. \\
    \emph{Ongoing submission towards TMLR}
    
    \item \textbf{Offense Detection in Dravidian Languages using Code-Mixing Index based Focal Loss and Cosine Normalization.} \href{https://link.springer.com/article/10.1007/s42979-022-01190-1}{\ExternalLink}\\
    \textbf{D Tula}, Shreyas Ms, V Reddy, P Sahu, S Doddapaneni, P Potluri, R Sukumaran, P Patwa.\\
    \emph{Springer Nature Computer Science, 2022 (Journal)}.
    
    \item \textbf{Ensemble of Multilingual Language Models with Pseudo Labeling for Offense Detection in Dravidian Languages.} \href{https://www.aclweb.org/anthology/2021.dravidianlangtech-1.42}{\ExternalLink}\\
    \textbf{D Tula}, P Potluri, Shreyas Ms, S Doddapaneni, P Sahu, R Sukumaran, P Patwa.\\
    \emph{DravidianLangTech @ European Association for Computational Linguistics (\textbf{EACL}), 2021}
    
    \item \textbf{Estimating RNA Secondary Structure by Maximizing Stacking Regions.} \href{https://doi.org/10.1007/978-981-15-6198-6_15}{\ExternalLink}\\
    P Sen, \textbf{D Tula}, S K Ray, S S Satapathy. \\
    \emph{International Conference on Computer Communication and Internet of Things (\textbf{ICCCIoT}), 2021.} \\
    \faTrophy \textcolor{black}{~\textbf{Best Paper Award}.}
    
    % \item \href{https://pubmed.ncbi.nlm.nih.gov/35920776/}{\textbf{Incorporation of transition to transversion ratio and nonsense mutations, improves the estimation of the number of synonymous and non-synonymous sites in codons.}}\\
    % Suvendra K Ray, Ruksana Aziz, Piyali Sen, Pratyush Kumar Beura, Saurav Das, \textbf{Debapriya Tula}, \\
    % Madhusmita Dash, Nima Dondu Namsa, Ramesh Chandra Deka, Edward J Feil, Siddhartha Sankar Satapathy.\\
    % \emph{DNA Research, 2022 (Journal)}.

\end{itemize}
\vspace{-1.5mm}

\section{Engineering Experience}
\textbf{Tata Consultancy Services - Innovation Lab, India} \href{https://www.tcs.com/what-we-do/research}{\ExternalLink} \hfill {\textit{Aug 2021 - July 2022}}\\
\emph{Machine Learning Engineer}
\begin{midlist}
    \item Built ML models for forecasting user health policy renewal on data (\textbf{$\sim$ 20 GB}) provided by General Electric HealthCare.
    \item Developed a framework to process diverse tabular data using AutoML toolkits and deployed it.
    \item Developed modules for statistical data analysis of the results for user interpretability using Plotly.
\end{midlist}
\vspace{2.5mm}

\textbf{LimeChat, India} \href{https://www.limechat.ai/}{\ExternalLink} \hfill {\textit{Jan 2021 - June 2021}}\\
\emph{NLP Software Development Intern}
\begin{midlist}
    \item Redesigned the \textbf{order tracking} system to make it more seamless and fault-tolerant (\textbf{30\%} reduction in user dropoffs).
    \item Redesigned LimeChat’s \textbf{FAQ} and \textbf{Utterance management} systems and deployed them as core features in \textbf{5 weeks}.
    \item Designed an end-to-end chatbot for \textbf{Nissan}, LimeChat’s biggest client undertaking hitherto.
\end{midlist}
\vspace{2.5mm}


\section{Select Projects}
\vspace{-4mm}
\label{sec:proj}
\begin{lonemidlist}
\item\textbf{Content-Based Image Retrieval} \hfill {\textit{Oct 2020 - May 2021}}
\begin{itemize}
    \item Developed a curriculum learning method for retrieving images from large datasets.
    \item Added a global attention module and an angular-based loss based soft to hard example sampler to help the model learn both simple and complex features. 
\end{itemize}
\item \textbf{Speech Emotion Recognition} \href{https://github.com/Debapriya-Tula/NLP-Project}{\ExternalLink} \hfill {\textit{Sep 2020 - Dec 2020}}
\begin{itemize}
     \item Applied augmentation to speech signals, extracted MFCC features and trained a Random Forest Classifier for identifying emotion from speech directly.
     \item Accuracy obtained on datasets: \href{https://journals.plos.org/plosone/article?id=10.1371/journal.pone.0196391} {RAVDESS}: \textbf{73.5 \%} \& \href{https://tspace.library.utoronto.ca/handle/1807/24487} {TESS}: \textbf{98.6 \%}.
\end{itemize}
% \item\textbf{Solingo} \href{https://github.com/Debapriya-Tula/Solve-it-in-a-go}{\ExternalLink} \hfill {\textit{Feb 2020 - Apr 2020}}
% \begin{itemize}
%      \item Developed an app that recognises handwritten math expressions from captured images.
%      \item Implemented an Densenet-based multiscale attention model for expression recognition, including LaTeX transformation of the input, using \textbf{Pytorch} and \textbf{Selenium}.
%      \item Accuracy on \href{https://researchdata.edu.au/crohme-competition-recognition-expressions-png/639782}{CROHME} dataset: \textbf{73 \%}.
% \end{itemize}
\item\textbf{Speech Dereverberation} \href{https://github.com/Debapriya-Tula/Speech_Dereverberation}{\ExternalLink}\hfill {\textit{Sep 2018 - Dec 2018}}
\begin{itemize}
     \item Implemented a statistical weighted prediction error model with a Gaussian prior over the reverb in a speech signal.
     \item Similarity with original signal: \textbf{65 - 75 \%}.
\end{itemize}
\end{lonemidlist}



\section{Awards and Honors}
\vspace{-4mm}
\begin{smallmidlist}
\item \textbf{Dept Rank 5} among \textbf{160} students - Computer Science and Engineering, IIIT Sri City.
\item \textbf{Best Paper Award} at \href{https://link.springer.com/book/10.1007/978-981-15-6198-6}{ICCCIoT}, 2020.
\item Awarded Innovation in Science Pursuit for Inspired Research (\href{https://online-inspire.gov.in/}{\textbf{INSPIRE}}) in 2013. 
\item \textbf{State rank 11} in National Science Talent Search Exam (\href{https://www.unifiedcouncil.com/about-nstse-online.html}{\textbf{NSTSE}}) in 2012.
\end{smallmidlist}
\vspace{1.0mm}


\section{Talks}
\vspace{-4mm}
\begin{lonemidlist}
\item \textbf{Deep Learning - Then, Now and Beyond} \href{https://docs.google.com/presentation/d/1B4fY_jALCVDi2TN13jeHtQngEngYqBdBN9k7VFwzvgE/edit#slide=id.p}{\ExternalLink}
\begin{itemize}
    \item Central University of Odisha, India \hfill{\textit{Apr 2023}}
\end{itemize}
\end{lonemidlist}


\section{Academic Service and Volunteering}
\vspace{-4mm}
\begin{lonemidlist}

\item \textit{Volunteer at Google Booth} - \textbf{ICCV, 2023}.
\item \textit{Reviewer} - \textbf{ICVGIP, 2024}; Workshops at \textbf{EACL 2021}, \textbf{NeurIPS 2022}, \textbf{ICCV 2023}.
\item \textit{AI Student Ambassador} - \textbf{Intel} \hfill {\textit{Oct 2019 - June 2021}}
  \begin{itemize}
        \item Organized 1 hr long hands-on sessions and paper reading sessions on topics related to AI/ML. Encouraged students to work on AI/ML projects and assisted them.
        \item Implemented a video frame interpolation method using adaptive separable convolutions for efficient internet data usage. Average interpolation error on \href{https://github.com/Debapriya-Tula/AdaConv-Pytorch}{Visual Tracker Benchmark(VTB)} dataset - \textbf{12.6 \%}.
  \end{itemize}
\item \textit{Undergraduate Teaching Assistantship} - Computer Science and Engineering, IIIT Sri City
  \begin{itemize}
        \item Advanced Data Structures and Algorithms - \href{https://profile.iiita.ac.in/srdubey/}{\textit{Prof. Shiv Ram Dubey}} \hfill{\textit{Fall 2019}}
        \item Data Structures and Algorithms - \href{https://mprerana.github.io/DrPreranaMukherjee/}{\textit{Prof. Prerana Mukherjee}} \hfill{\textit{Spring 2020}}
  \end{itemize}
\end{lonemidlist}
\vspace{1.0mm}

\section{Relevant Coursework}
\textbf{Math} - Discrete Mathematics, \textit{Linear Algebra}, Probability Theory, Statistical Data Analysis, Advanced Statistical Methods \\
\textbf{Computer Science} - Theory of Computation, \textit{Artificial Intelligence}, Digital Image Processing, Computer Graphics and Multimedia, \textit{Deep Learning}, \textit{Computer Vision}, Natural Language Processing
\vspace{1.0mm}

\section{Languages and Tools}
\textbf{Python}, MATLAB, Javascript, \textbf{Git}, SQL, NoSQL, \textbf{Bash}, Rasa, \textbf{Tensorflow}, \textbf{Pytorch}, \textbf{Keras}, \textbf{JAX}, FastAI, Sklearn, \textbf{Numpy}, \textbf{Pandas}, Seaborn, LaTeX
\vspace{1.0mm}

% \section{Other Details}
% \vspace{-4mm}
% \begin{lonemidlist}

% \item \textbf{Language Proficiency:} English, Hindi, Odia, Assamese.
% \item \textbf{Hobbies:} Reading books, playing the guitar, singing, playing table tennis and badminton.
% \end{lonemidlist}

\end{document}

%%%%%%%%%%%%%%%%%%%%%%%%%% End CV Document %%%%%%%%%%%%%%%%%%%%%%%%%%%%%

%----------------------------------------------------------------------%
% The following is copyright and licensing information for
% redistribution of this LaTeX source code; it also includes a liability
% statement. If this source code is not being redistributed to others,
% it may be omitted. It has no effect on the function of the above code.
%----------------------------------------------------------------------%, .
% Copyright (c) 2007, 2008, 2009, 2010, 2011 by Theodore P. Pavlic
% Copyright (c) 2020 by Aditya Kusupati
%
% Unless otherwise expressly stated, this work is licensed under the
% Creative Commons Attribution-Noncommercial 3.0 United States License. To
% view a copy of this license, visit
% http://creativecommons.org/licenses/by-nc/3.0/us/ or send a letter to
% Creative Commons, 171 Second Street, Suite 300, San Francisco,
% California, 94105, USA.
%
% THE SOFTWARE IS PROVIDED "AS IS", WITHOUT WARRANTY OF ANY KIND, EXPRESS
% OR IMPLIED, INCLUDING BUT NOT LIMITED TO THE WARRANTIES OF
% MERCHANTABILITY, FITNESS FOR A PARTICULAR PURPOSE AND NONINFRINGEMENT.
% IN NO EVENT SHALL THE AUTHORS OR COPYRIGHT HOLDERS BE LIABLE FOR ANY
% CLAIM, DAMAGES OR OTHER LIABILITY, WHETHER IN AN ACTION OF CONTRACT,
% TORT OR OTHERWISE, ARISING FROM, OUT OF OR IN CONNECTION WITH THE
% SOFTWARE OR THE USE OR OTHER DEALINGS IN THE SOFTWARE.
%----------------------------------------------------------------------%
